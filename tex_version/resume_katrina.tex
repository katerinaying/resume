\documentclass[10pt,a4paper,sans]{moderncv}        % possible options include font size ('10pt', '11pt' and '12pt'), paper size ('a4paper', 'letterpaper', 'a5paper', 'legalpaper', 'executivepaper' and 'landscape') and font family ('sans' and 'roman')

% moderncv themes
\moderncvstyle{classic}                             % style options are 'casual' (default), 'classic', 'oldstyle' and 'banking'
\moderncvcolor{blue}                               % color options 'blue' (default), 'orange', 'green', 'red', 'purple', 'grey' and 'black'
%\renewcommand{\familydefault}{\sfdefault}         % to set the default font; use '\sfdefault' for the default sans serif font, '\rmdefault' for the default roman one, or any tex font name
%\nopagenumbers{}                                  % uncomment to suppress automatic page numbering for CVs longer than one page
                        % if you need to use CJK to typeset your resume in Chinese, Japanese or Korean

% adjust the page margins
\usepackage[scale=0.9]{geometry}
%\setlength{\hintscolumnwidth}{3cm}                % if you want to change the width of the column with the dates
%\setlength{\makecvtitlenamewidth}{10cm}           % for the 'classic' style, if you want to force the width allocated to your name and avoid line breaks. be careful though, the length is normally calculated to avoid any overlap with your personal info; use this at your own typographical risks...

% personal data
\name{Katrina(Ying)}{Zhang}
\title{Senior Software Engineer}                               % optional, remove / comment the line if not wanted
\phone[mobile]{+86~(136)~7152~7121}                   % optional, remove / comment the line if not wanted; the optional "type" of the phone can be
\email{katerinaying@gmail.com}                               % optional, remove / comment the line if not wanted
\social[linkedin]{katrina-zhang}                        % optional, remove / comment the line if not wanted
\photo[64pt][0.4pt]{pic/katrina}                       % optional, remove / comment the line if not wanted; '64pt' is the height the picture must be resized to, 0.4pt is the thickness of the frame around it (put it to 0pt for no frame) and 'picture' is the name of the picture file
%\quote{Some quote}                                 % optional, remove / comment the line if not wanted


%----------------------------------------------------------------------------------
%            content
%----------------------------------------------------------------------------------
\begin{document}
%-----       resume       ---------------------------------------------------------
\makecvtitle

\section{Summary}
%TODO: add more keywords
\cvitem{}{3 years software engineer of distributed concurrent system on Linux Platform }

\section{Education}
\cventry{2008-2012}{Bachelor Degree}{East China Normal University(ECNU)}{Shanghai}{Communication Engineering}{}

\section{Vocational Experience}
\cventry{2012.7-Present}{Senior Softare Engineer}{Ericsson}{Shanghai}{}{%
Daily work includes:%
\begin{itemize}%
\item \textbf{Distributed performance testing product}
    \begin{itemize}
    \item This is an internal testing product used to provide stability, robustness, capacity testing to different commercial products.
    \item Design and implement a quick prototype to fulfill customers' 5G testing requirement.
    \item Investigate and tune performance of product, reducing 20 percent of memory consumption.
    \item Handle tricky problems & urgent tasks from customers.
    \item Support team to achieve a good design during feature development & product maintence, responsible for the qualify of product.
    \item Propose techinical optimization ideas/advice to product owner.
    \item Drive technical competence buildup and knowlege sharing.
    \item \textbf{Techical stack}
        \begin{itemize}
        \item Erlang, C++, Perl, RHEL6
        \end{itemize}
    \item \textbf{Tools}
        \begin{itemize}
        \item Git, EMACS, IntelliJ
        \end{itemize}
    \end{itemize}
\end{itemize}}

\begin{itemize}%
\item \textbf{Test log evaluation tool}
    \begin{itemize}
    \item This is an analyzer aimed to find the root cause of failure through parsing logs generated by testcases.
    \item Design and implement by cooperating with cross national customers and developers.
    \item \textbf{Techical stack}
        \begin{itemize}
        \item Java, Bash, Perl
        \end{itemize}
    \item \textbf{Tools}
        \begin{itemize}
        \item Eclipse, Vim
        \end{itemize}
    \end{itemize}
\end{itemize}}

\begin{itemize}%
\item \textbf{Wireshark}
    \begin{itemize}
    \item Develop new protocol parsers for internal customers, working on both Windows & Linux platform.
    \item Handle TRs reported by customers, contributing the fix to wireshark community.
    \item \textbf{Techical stack}
        \begin{itemize}
        \item C, Bash
        \end{itemize}
    \item \textbf{Tools}
        \begin{itemize}
        \item Source Insight, TortoiseGit
        \end{itemize}
    \end{itemize}
\end{itemize}}

\section{Languages}
\cvitem{English}{proficiency, CET6 certificate}

\clearpage

\end{document}
